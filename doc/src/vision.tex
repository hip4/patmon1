\documentclass[fontsize=12pt,
               paper=a4,
               twoside=false,
               parskip=half,
               ]{scrartcl}

% Load the packages
\input{./packages.tex}

% SVN Meta
\SVN $Date: 2012-05-04 16:15:57 +0200 (Fr, 04 Mai 2012) $
\SVN $Revision: 98 $
\SVN $HeadURL: https://svn.bfh.ch/repos/projects/patmon1/trunk/doc/src/vision.tex $


\begin{document}

% Document title for title.tex
\newcommand{\doctitle}{Vision}
\input{./title.tex}

\tableofcontents

\section{Introduction}

We envision a flexible and easy to administrate application devoted to health monitoring of individuals or patients. The application shall support a large number of patients, many doctors can use the system simultaneously, and a single administrator administrates the system. The administrator is not a doctor. Usually doctors work in the same building and share some devices for the health monitoring that are stored in a given room.

\subsection{Problem Statement}

A company called \emph{Orbital Tech Inc.} has developed an electronic device called \emph{OT-Logger} that allows doctors to monitor the temperature of some of their patients remotely. This device measures the body temperature of a patient on a regular basis and transmits the measure together with date and time. The device is attached to the patient during a given observation period, which lasts from a few days to a few weeks. The patient carries such a device 24 hours, can go home and go to work while the device operates. After an observation period ends, the device is returned and can be assigned to another patient.

A simple stand-alone application is needed to store the measures performed by various devices assigned to the patients. Such an application allows a doctor to consult the temperature data of a patient he/she treats.

The administrator is in charge of creating an account for each doctor because many doctors can use the application. The password is automatically generated by the system, which is handed over to the doctor in person by the administrator. The system also sends automatically an email to the doctor via an email server. This email contains an activation code (also generated automatically) that is used only once during the authentication phase. During the authentication phase the system detects that the account in not activated yet. The doctor must first activate it by entering it's activation code. If the account is already activated the activation process is skipped. Once activated the user must authenticate himself/herself by means of its email address (identifier) and its personal password.

A doctor records each patient only once and, for a sake of simplicity, a patient is treated by only one doctor. A doctor can query the system to determine all the observation periods for a given patient that he/she treats.
For each observation period of a patient the doctor must enter the following data:

\begin{enumerate}
\item Which device is assigned to this patient. The system verifies that the device is not already in use.
\item The beginning and the end of the desired observation period.
\item The frequency by which the measures have to be performed.
\end{enumerate}

The system verifies that all the provided values are coherent, in particular that the device is not already in use.
After this the system initializes the device assigned to the patient with the following data:

\begin{enumerate}
\item Beginning and end of the observation period.
\item Frequency of measures.
\end{enumerate}

The application collects the measures by means of a \enquote{polling} technique, that is to say the application queries the remote device on a regular basis to obtain the last performed measure. Note that the device does not store any patient identifier.

The system should run on Unix/GNU-Linux as well as on Microsoft Windows.

\section{Stakeholder Descriptions}

\subsection{Doctor}

Doctors use the system for taking measurements on patients. They are interested in a simple interface for collecting their patient's data. The system has to be reliable and error proof.

\subsection{Administrator}

The administrator configures the system for the use by the doctors. He is interested in an easy, error proof and secure system interface.

\subsection{Patients}

The patient does not want to spent too much time waiting when his device is configured. He is also interested in the reliability of the system as he does not want to do an observation twice because of a system failure.

\subsection{Orbital Tech}

Orbital Tech, the manufacturer of the monitoring device, OT-Logger, is interested in the success of the project, because it profits from the operation of the monitoring system which uses it's devices.

\subsection{Client}

The client is a big hospital which needs a system for collecting data from the monitoring devices. The hospital is led by Prof. Dr. Olivier Biberstein, which represents the client as the single point of contact to the project team.

\subsection{Project team}

The project team (aka. Comet Engineering) consists of Patrick Haring and Christian Bürgi.

\end{document}